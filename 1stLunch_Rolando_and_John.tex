\documentclass{article}

\usepackage{amsmath, amsthm, amssymb,xcolor,url,booktabs,physics}  %cite
\urlstyle{sf}
\usepackage{graphicx}
\usepackage[margin=1in]{geometry}
\synctex=1

\ifdefined\blackandwhite
\usepackage[pdftex]{hyperref} %pagebackref
\else
\usepackage[colorlinks,pdftex]{hyperref} %pagebackref
\hypersetup{citecolor = blue, linkcolor = red!70!black}
\fi
\usepackage[linewidth=1pt]{mdframed}

\newcommand{\nc}{\newcommand}
\nc{\rnc}{\renewcommand}
\renewcommand{\ket}[1]{|{#1}\rangle}
\newcommand{\kett}[1]{|{#1}\rrangle}
\renewcommand{\bra}[1]{\langle{#1}|}
\newcommand{\braa}[1]{\llangle{#1}|}
\renewcommand{\braket}[2]{\langle{#1}|{#2}\rangle}
\newcommand{\brakett}[2]{\llangle{#1}|{#2}\rrangle}
\renewcommand{\norm}[1]{\|{#1}\|}
\newcommand{\Norm}[1]{\left\|{#1}\right\|}




\DeclareMathOperator{\Bin}{Bin}
\DeclareMathOperator{\conv}{conv}
\DeclareMathOperator{\id}{id}
\DeclareMathOperator{\Img}{Im}
\DeclareMathOperator{\inj}{inj}
\DeclareMathOperator{\Par}{Par}
\DeclareMathOperator{\poly}{poly}
\DeclareMathOperator{\polylog}{polylog}
\DeclareMathOperator{\sgn}{sgn}
\DeclareMathOperator{\Sep}{Sep}
\DeclareMathOperator{\SepSym}{SepSym}
\DeclareMathOperator{\Span}{span}
\DeclareMathOperator{\supp}{supp}
\DeclareMathOperator{\swap}{SWAP}
\DeclareMathOperator{\Sym}{Sym}
\DeclareMathOperator{\ProdSym}{ProdSym}
\DeclareMathOperator{\SEP}{SEP}
\DeclareMathOperator{\PPT}{PPT}
\DeclareMathOperator{\Var}{Var}
\DeclareMathOperator{\Wg}{Wg}
\DeclareMathOperator{\WMEM}{WMEM}
\DeclareMathOperator{\WOPT}{WOPT}

\DeclareMathOperator{\BPP}{\mathsf{BPP}}
\DeclareMathOperator{\BQP}{\mathsf{BQP}}
\DeclareMathOperator{\cnot}{\normalfont\textsc{cnot}}
\DeclareMathOperator{\DTIME}{\mathsf{DTIME}}
\DeclareMathOperator{\NTIME}{\mathsf{NTIME}}
\DeclareMathOperator{\MA}{\mathsf{MA}}
\DeclareMathOperator{\NP}{\mathsf{NP}}
\DeclareMathOperator{\coNP}{\mathsf{co-NP}}
\DeclareMathOperator{\PH}{\mathsf{PH}}
\DeclareMathOperator{\SharpP}{\mathsf{\#P}}
\DeclareMathOperator{\NEXP}{\mathsf{NEXP}}
\DeclareMathOperator{\Ptime}{\mathsf{P}}
\DeclareMathOperator{\QMA}{\mathsf{QMA}}
\DeclareMathOperator{\QCMA}{\mathsf{QCMA}}
\DeclareMathOperator{\BellQMA}{\mathsf{BellQMA}}
\DeclareMathOperator{\PostBPP}{\mathsf{PostBPP}}
\DeclareMathOperator{\PostBQP}{\mathsf{PostBQP}}

\def\sat{\text{3-SAT}}
\def\be#1\ee{\begin{equation}#1\end{equation}}
\def\bea#1\eea{\begin{eqnarray}#1\end{eqnarray}}
\def\beas#1\eeas{\begin{eqnarray*}#1\end{eqnarray*}}
\def\ba#1\ea{\begin{align}#1\end{align}}
\def\bas#1\eas{\begin{align*}#1\end{align*}}
\def\bpm#1\epm{\begin{pmatrix}#1\end{pmatrix}}
\nc{\non}{\nonumber}
\nc{\nn}{\nonumber}
\nc{\eq}[1]{(\ref{eq:#1})}
\nc{\eqs}[2]{(\ref{eq:#1}) and (\ref{eq:#2})}
%\def\eq#1{Eq.~(\ref{eq:#1})}
%\def\eqs#1#2{Eqs.~(\ref{eq:#1}) and (\ref{eq:#2})}
\rnc{\L}{\left} 
\nc{\R}{\right}
\nc{\ra}{\rightarrow}
\nc{\ot}{\otimes}


\newtheorem{thm}{Theorem}
\newtheorem*{thm*}{Theorem}
\newtheorem{claim}[thm]{Claim}
\newtheorem{cor}[thm]{Corollary}
\newtheorem{lem}[thm]{Lemma}
\newtheorem{prop}[thm]{Proposition}
\newtheorem{proto}{Protocol}
\newtheorem{con}[thm]{Conjecture}
\theoremstyle{definition}
\newtheorem{remark}{Remark}
\newtheorem{observation}{Observation}
\newtheorem{example}{Example}
\newtheorem{conjecture}{Conjecture}
\newtheorem{dfn}[thm]{Definition}
\theoremstyle{plain}

\makeatletter
\newtheorem*{rep@theorem}{\rep@title}
\newcommand{\newreptheorem}[2]{%
\newenvironment{rep#1}[1]{%
 \def\rep@title{#2 \ref{##1} (restatement)}%
 \begin{rep@theorem}}%
 {\end{rep@theorem}}}
\makeatother

\newreptheorem{thm}{Theorem}
\newreptheorem{lem}{Lemma}


\nc\eps{\epsilon}

\def\vb{\vec{b}}
\def\vx{\vec{x}}
\def\Usch{U_{\text{Sch}}}

\nc\cA{\mathcal{A}}
\nc\cB{\mathcal{B}}
\nc\cC{\mathcal{C}}
\nc\cD{\mathcal{D}}
\nc\cE{\mathcal{E}}
\nc\cF{\mathcal{F}}
\nc\cG{\mathcal{G}}
\nc\cH{\mathcal{H}}
\nc\cI{\mathcal{I}}
\nc\cJ{\mathcal{J}}
\nc\cK{\mathcal{K}}
\nc\cL{\mathcal{L}}
\nc\cM{\mathcal{M}}
\nc\cN{\mathcal{N}}
\nc\cO{\mathcal{O}}
\nc\cP{\mathcal{P}}
\nc\cQ{\mathcal{Q}}
\nc\cR{\mathcal{R}}
\nc\cS{\mathcal{S}}
\nc\cT{\mathcal{T}}
\nc\cU{\mathcal{U}}
\nc\cV{\mathcal{V}}
\nc\cW{\mathcal{W}}
\nc\cX{\mathcal{X}}
\nc\cY{\mathcal{Y}}
\nc\cZ{\mathcal{Z}}

\def\bp{\mathbf{p}}
\def\bq{\mathbf{q}}
\def\bP{{\bf P}}
\def\bQ{{\bf Q}}
\def\gl{\mathfrak{gl}}

\nc\bbC{\mathbb{C}}
\DeclareMathOperator*{\E}{\mathbb{E}}
\DeclareMathOperator*{\bbE}{\mathbb{E}}
\nc\bbF{\mathbb{F}}
\nc\bbM{\mathbb{M}}
\nc\bbN{\mathbb{N}}
\nc\bbR{\mathbb{R}}
\nc\bbZ{\mathbb{Z}}

\nc\benum{\begin{enumerate}}
\nc\eenum{\end{enumerate}}
\nc\bit{\begin{itemize}}
\nc\eit{\end{itemize}}
\newcommand{\fig}[1]{Fig.~\ref{fig:#1}}
\newcommand{\secref}[1]{Section~\ref{sec:#1}}
\newcommand{\appref}[1]{Appendix~\ref{sec:#1}}
\newcommand{\lemref}[1]{Lemma~\ref{lem:#1}}
\newcommand{\thmref}[1]{Theorem~\ref{thm:#1}}
\newcommand{\propref}[1]{Proposition~\ref{prop:#1}}
\newcommand{\protoref}[1]{Protocol~\ref{proto:#1}}
\newcommand{\defref}[1]{Definition~\ref{def:#1}}
\newcommand{\corref}[1]{Corollary~\ref{cor:#1}}
\newcommand{\conref}[1]{Conjecture~\ref{con:#1}}
\newcommand{\tabref}[1]{Table~\ref{tab:#1}}

\newcommand{\FIXME}[1]{{\color{red}FIXME: #1}}
\nc{\todo}[1]{\textcolor{red}{todo: #1}}
\nc{\Anote}[1]{\textcolor{red}{Aram note: #1}}

\newcommand{\boxdfn}[2]{
\begin{figure}[h]\begin{center}
\noindent \framebox{\begin{minipage}{0.8\textwidth}
\begin{dfn}[{\bf #1}]\ \\ \\#2
\end{dfn}\end{minipage}}
\end{center}\end{figure}}

\newcommand{\boxproto}[2]{
\begin{figure}[h]\begin{center}
\noindent \framebox{\begin{minipage}{0.8\textwidth}
\begin{proto}[{\bf #1}]\ \\ \\#2\end{proto}
\end{minipage}}
\end{center}\end{figure}}

\def\begsub#1#2\endsub{\begin{subequations}\label{eq:#1}\begin{align}#2\end{align}\end{subequations}}
\nc\mnb[1]{\medskip\noindent{\bf #1}}
\newcommand{\iso}[1]{\stackrel{#1}{\cong}}
\nc{\pder}[2]{\frac{\partial {#1}}{\partial {#2}}}
\nc{\p}{\partial}

\newcounter{lecnum}

\newcommand{\lecture}[4]{
   \pagestyle{myheadings}
   \renewcommand{\thepage}{\thelecnum-\arabic{page}}
   \renewcommand{\thesection}{\thelecnum.\arabic{section}}
   \renewcommand{\theequation}{\thelecnum.\arabic{equation}}
   \renewcommand{\thefigure}{\thelecnum.\arabic{figure}}
   \renewcommand{\thetable}{\thelecnum.\arabic{table}}
   \thispagestyle{plain}
   \newpage
   \setcounter{lecnum}{#1}
   \setcounter{page}{1}
   \noindent
   \begin{center}
   \framebox{
      \vbox{\vspace{2mm}
    \hbox to 6.28in { {\bf %more details 
     } }
       \vspace{4mm}
       \hbox to 6.28in { {\Large \hfill QLunch \# #1: #2  \hfill} }
       \vspace{2mm}
       \hbox to 6.28in { {\it Speakers: #3 \hfill Scribe: #4} }
      \vspace{2mm}}
   }
   \end{center}
   \markboth{Q-Lunch #1: #2}{QLunch #1: #2}
}


%%%%%%%%%%%%%%%%%%%%
\begin{document}
\lecture{1}{April 12, 2017}{Rolando  La Placa- John Napp}{Mehdi Soleimanifar}


\paragraph{Topics:}{First, Rolando talked about the universal recovery map for approximate Markov chains \cite{FR15}, and then John told us about the equivalence of the quantum approximate Markov chains and Gibbs states of one-dimensional local quantum Hamiltonians \cite{KB16}.}

 
\section{Universal recovery map for approximate Markov chains \cite{FR15}}
  \begin{mdframed}
 \textbf{tl;dr:} given a state $\rho_{ABC}$, if the conditional mutual information is small, $A$ is only correlated to $C$ through $B$ up to a small error, in the sense that $C$ can be approximately recovered given the information contained in $B$ only. These states are called \textit{approximate quantum Markov chain}. 
\end{mdframed}
\subsection{Classical case}
In classical case, we have a clear relation between \textit{conditional independence} of two random variables and \textit{conditional mutual information}. In particular, we have

\begin{thm}
Two random variables $A$ and $B$ are conditionally independent given $B$, meaning $\forall a,b,c$,
\begin{align}\label{eq:1}
p_{ABC}(a,b,c) = p_A(a)P_{B|A}(b|a)p_{C|B}(c|b),
\end{align}
if and only if
\begin{align}
	I(A:C|B)=0.
\end{align} 
 \end{thm}
We call such chain of variables, $A-B-C$, an exact Markov chain. \\

According to (\ref{eq:1}), for each Markov chain with a fixed conditional distribution $p_{C|B}(c|b)=\mathcal{N}(c|b)$, we have 
\begin{align*}
	p_{ABC}(a,b,c)=N(c|b)p_{AB}(a,b),
\end{align*} 
 so determining $p_{AB}$ and $a,b,c$ is enough to get $p_{ABC}(a,b,c)$. In other words, there exists a \textit{recovery map} $\mathcal{R}^{\mathcal{N}}(c|b)$ that maps $p_{AB}(a,b)$ to $p_{ABC}(a,b,c)$.\\
 \subsection{Quantum case}
 In quantum case, however, things are not as obvious as before, and the notion of conditional independence is not clear at first glance. Starting with a state $\rho_{ABC}$ on a tripartite quantum system $A\otimes B \otimes C$, we define the conditional mutual information 
 \begin{align} 
 I(A:C|B)=H(AB)+ H(BC)- H(B) - H(ABC). 
 \end{align}
Then, we have the non-trivial \textit{strong subadditivity} property which says $I(A,C|B)\geq 0$. It has been shown \cite{FR15} that the states with $I(A,C|B)=0$ characterize states ${\rho}_{ABC}$ whose system $C$ can be reconstructed just by acting on $B$, i.e., as in the classical case, there exists a recovery map $\mathcal{R}_{A\rightarrow BC}$ from $B$ to $B \otimes C$ such that
\begin{align}
	\rho_{ABC}=\mathcal{R}_{A\rightarrow BC}(\rho_{AB}).
\end{align}
We call these states (exact) quantum Markov chains. 
\footnote{The specific form of the recovery map is also known, and is of the form $X_B  \mapsto \rho_{BC}^{\frac{1}{2}}(\rho_B^{-\frac{1}{2}}X_B \rho_B^{-\frac{1}{2}}\otimes \id_C)\rho_{BC}^{\frac{1}{2}}$.}
To get some intuition, consider the case when $B$ is classical, i.e. 
\begin{align}
	 \rho_{ABC}=\sum_b p_B(b) \ketbra{b}{b}_B \otimes \rho_{AC,b}
\end{align}
When $\rho_{BC}=\rho_{B}\otimes \rho_{C}$, we expect this state to be Markovian since conditioned on the value of $B$, the marginal state $\rho_{AC,b}$ is a product state, meaning the two systems $A$ and $B$ are independent -- as required by our definition of a Markov chain. One can easily see that in this case indeed we have $I(A;B|C)=0$. Also, there exists a recovery map $R_{B\rightarrow BC}$ such that 
\begin{align}
	\mathcal{R}_{B\rightarrow BC} (\ketbra{b}{b}) = \ketbra{b}{b} \otimes  \rho_{C,b}  \quad (\forall b),
\end{align}
where $\rho_{C,b}=\tr_C(\rho_{AC,b})$. 
This observation can be generalized as shown in \cite{FR15} to the following fact
\begin{thm}
	For a state $\rho_{ABC}$ satisfying strong subadditivity with equality, i.e. $I(A : C|B)=0$, the marginal state $\rho_{AC}$ is separable. Conversely, for each separable state $\rho_{AC}$, there exists an extension $ρ\rho_{ABC}$ such that $I(A : C|B) = 0$.
\end{thm}
Thus, in the exact case of $I(A:C|B)=0$, we have a meaningful generalization of the conditional independence in terms of recovery maps, but what about the approximate case where $I(A:C|B)\leq \varepsilon$? It turns out \cite{HJPW04} that we can still have the same interpretation:
\begin{thm}
	For any density operator $\rho_{BC}$ on $B\otimes C$ there exists a trace-preserving completely positive map $\mathcal{R}_{B\rightarrow BC}$ such that for any extension $\rho_{ABC}$ on $A \otimes B \otimes C$ 
	\begin{align}\label{eq:2}
		I(A : C|B) \geq -2\log_2 F(\rho_{ABC},\mathcal{R}_{B\rightarrow BC}(\rho_{AB}),
	\end{align}
	or equivalently 
	\begin{align}
		F(\rho_{ABC},\mathcal{R}_{B\rightarrow BC}(\rho_{AB}))\geq 2^{-\frac{1}{2}I(A:C|B)}.
	\end{align}
	where $F(.,.)$ is the fidelity of the states defined as $F(\rho,\sigma)=\Norm{\sqrt{\rho}\sqrt{\sigma}}_1$.
\end{thm}
Notice that (\ref{eq:2}) immediately gives both the strong subadditivity inequality, and the existence of the recovery map in the exact Markov chain discussed before. Also,  the recovery map is \textit{universal} in the sense that the result holds if the recovery map is chosen depending on $\rho_{BC}$ only, rather than on $\rho_{ABC}$.\\


\section{Equivalence of the quantum approximate Markov chains and Gibbs states}

We can extend the definition of Markov chain to an $n$-partite system and say $\rho_{A_1,\dots A_n}$ is a quantum $\varepsilon$-approximate Markov chain if 
\begin{align}
	I(A_i,\dots,A_{i-1}:A_{i+1},\dots ,A_n|A_i)\leq \varepsilon \quad i\in [n]
\end{align}

\subsection{Classical case}
Classically, Markov chains are equivalent to the set of Gibbs states of 2-local Hamiltonians:
\begin{align}
	p_{A_1,\dots A_n}(a_1,\dots a_n)=\frac{\exp{-\sum_i h_i(a_i,a_{i+1})}}{Z}
\end{align}
This result is called Hammersley-Clifford Theorem. 
\subsection{Quantum case}
It has been shown that the same theorem remains valid in the case of an exact quantum Markov chain. Namely, a full rank quantum state $\rho _{A_1,\dots,A_n}$ is a quantum Markov chain if, and only if, it can be written as
\begin{align}
	\rho=\frac{\exp(-\sum_i h_i(a_i,a_{i+1}))}{Z}
\end{align}
where each $h_{i,i+1}$ only acts on subsystems $A_i, A_{i+1}$, such that $[h_{i,i+1}, h_{j,j+1}] = 0$ for all $i, j$. Therefore we have a characterization of full rank quantum Markov chains as Gibbs states of 1D commuting local quantum Hamiltonians.
The main result of \cite{KB16} is that if we relax the condition of the previous result from exact Markov chains to approximate case, we get the following:
\begin{itemize}
	\item[-] Let $H=\sum_i h_i$ be a short-range one-dimensional Hamiltonian with $\Norm{h_i} \leq 1$. Then for every tripartite split of the system $ABC$,  if we choose the region $B$ sufficiently large, the Gibbs state can be approximately recovered from the partial trace over $C$ by performing a recovery map on $B$. In turn, this  implies the conditional mutual information between two regions $A$ and $C$ given a middle region $B$ decays exponentially with the square root of the length of $B$. 
	\item[-]  Conversely, let $\rho _{A_1,\dots,A_n}$ be an quantum $\varepsilon$-approximate Markov chain. Then there exists a local Hamiltonian $H=\sum_i h_{A_,A_{i+1}}$, such that
	\begin{align}
		S(\rho ||\frac{e^{-H}}{\tr(e^{-H})})\leq \varepsilon n
	\end{align}
\end{itemize}
the combination of the two results gives as a variant of the Hammersley-Clifford theorem for quantum approximate Markov chains.

  \begin{thebibliography}{99}

\bibitem{FR15} O. Fawzi and R. Renner. \textit{Quantum conditional mutual information and approximate markov chains}. Communications in Mathematical Physics, 340(2):575–611, 2015. \href{https://arxiv.org/pdf/1504.07251.pdf}{https://arxiv.org/pdf/1504.07251.pdf}


\bibitem{HJPW04} P. Hayden, R. Jozsa, D. Petz, and A. Winter. \textit{Structure of states which satisfy strong subadditivity of quantum entropy with equality.} Communications in Mathematical Physics, 246(2):359–374, 2004. \href{https://arxiv.org/pdf/quant-ph/0304007.pdf}{https://arxiv.org/pdf/quant-ph/0304007.pdf}

\bibitem{KB16} Kohtaro Kato and F.G.S.L. Brandao. \textit{Quantum approximate markov chains are thermal}. arXiv preprint: \href{https://arxiv.org/pdf/1609.06636.pdf}{https://arxiv.org/pdf/1609.06636.pdf}

\end{thebibliography}
\end{document} 
